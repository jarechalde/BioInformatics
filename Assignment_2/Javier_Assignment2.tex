\documentclass{article}
\usepackage{listings}
\usepackage{amsmath}
\usepackage{graphicx}
\usepackage{float}
\usepackage{subcaption}
\usepackage[linewidth=1pt]{mdframed}
\usepackage[colorlinks]{hyperref}

\usepackage{algorithm}
\usepackage{algpseudocode}

\usepackage{verbatim}

\hypersetup{citecolor=DeepPink4}
\hypersetup{linkcolor=DarkRed}
\hypersetup{urlcolor=Blue}

\usepackage{cleveref}

\setlength{\parindent}{1em}
\setlength{\parskip}{1em}
\renewcommand{\baselinestretch}{1.0}

\begin{document}

\begin{titlepage}
	\centering
	{\scshape\LARGE Assignment 2\par}
	\vspace{1cm}
	{\scshape\Large Bioinformatics (CIS 455)\par}
	\vspace{1.5cm}
	{\Large\itshape Javier Arechalde\par}
	\vfill
	{\large \today\par}
\end{titlepage}

\section*{Problem 2-1 Jones \& Pevzner, Problem 4.1}

In this problem, we will be given a set $X$, and we will need to calculate the multiset $\Delta X$,

\subsection*{Pseudocode}

In our implementation, we will iterate over the list with two index, the first one to indicate the position we are calculating the distances from, and the second one to indicate the position we want to calculate the distance with. We suppose that the given list comes presorted, so we won't.

\begin{algorithm}[H]
\caption{title}
\begin{algorithmic}
\For{$i$ in range $1\to length(X)$}
 \For{$j$ in range $j\to length(X)$}
  \State $distance = X[j]-X[i]$
  \State We add the calculated distance to the $\Delta X$ array
 \EndFor
\EndFor
\end{algorithmic}
\end{algorithm}

\subsection*{Implementation}

Here is the implementation of the pseudocode listed above.

\lstinputlisting[language=Python]{Problem21.py}

After running the code over a sample dataset, we obtained the following results.

\verbatiminput{OutputProblem21.txt}

\section*{Problem 2-2 Jones \& Pevzner, Problem 4.2}

\subsection*{Following $ANOTHERBRUTEFORCEPDP(L,n)$}

Here is the implementation code for $ANOTHERBRUTEFORCEPDP$, note that for finding all the possible sets of length $n-2$, we decided to use \textit{itertools}, to make the implementation easier.

\lstinputlisting[language=Python]{Problem22.py}

After running the code over the given partial digest, we obtained the following results.

\verbatiminput{OutputProblem22.txt}

\section*{Problem 2-3 Jones \& Pevzner, Problem 4.5}

Two sets $A$ and $B$ are said to be homometric if $\Delta A = \Delta B$.

We will try to prove that

$$U\oplus V = \{ u+v:u\in U,v\in V \}$$

and

$$U\ominus V = \{ u-v:u\in U,v\in V \}$$

are homometric.

In this problem we have two sets:

$$U = \{u_1,u_2,...,u_n\}$$
$$V = \{v_1,v_2,...,v_n\}$$

Then,

$$A = U\oplus V = \{ u_1+v_1,u_1+v_2,...,u_1+v_n,...,u_n+v_1,u_n+v_2,...u_n+v_n\}$$

$$B = U\ominus V = \{ u_1-v_1,u_1-v_2,...u_1-v_n,...,u_n-v_1,u_n-v_2,...u_n-v_n\}$$

Now we calculate $\Delta A$ and $\Delta B$:

$$\Delta A = \{ u_1+v_1 - (u_1+v_2),...\} = \{v_1-v_2,...\}$$
$$\Delta B = \{ u_1-v_1 - (u_1-v_2),...\} = \{v_2-v_1,...\}$$

As we can see, the elements from $U$ disappear when we are calculating $\Delta A$ and $\Delta B$, and then, we are only calculating the distances within the set $U$. And as the distance is an absolute value of the difference of the two points, $v_1-v_2 = v_2 - v_1$, and this is true for each one of the elements int $\Delta A$ and $\Delta B$. 

Then, we proven that $\Delta A = \Delta B$ and thus, sets $U\oplus V$ and $U \ominus V$ are \textit{homometric}.

\section*{Problem 2-4 Jones \& Pevzner, Problem 4.9}

\subsection*{Pseudocode}

In this problem we are given three sets of fragment lengths $A$, $B$ and $A+B$. The goal is to reconstruct the original restriction map for each one of the enzymes, $A$ and $B$.

For our brute force implementation, we will use all the possible orderings of the different fragments from $A$, $B$, and $A+B$ and see if they "match".

\begin{algorithm}[H]
\caption{DDP Brute Force}
\begin{algorithmic}[1]
\For{Every possible ordering of $A+B$}
 \For{Every possible ordering of $A$}
  \For{Every possible ordering of $B$}
   \For{fragment $dX(i)$ in $dX$}
    \If{$dX(i) == A[j]$}
     \State Append to the A position $A[j] = sum(dX[0:i])$
     \State j++
    \ElsIf{$dx{i} == B[k]$}
     \State Append to the A position $B[k] = sum(dX[0:i])$
     \State k++
    \Else{It doesnt fit any of the fragments}
     \State Continue to the next iteration
    \EndIf
   \EndFor
  \EndFor
 \EndFor
\EndFor
\end{algorithmic}
\end{algorithm}

\subsection*{Branch and Bound}

A \textit{Branch and Bound} implementation for this algorithm could list all the possible combinations of $A+B$ as a tree, having as the root nodes only the nodes that are equal or smaller to the first fragments of $A$ and $B$. While we explored the tree, we could check if the sequence that we are building matches the other fragment lengths in $A$ and $B$, and if it doesnt, we could skip that branch, and go to the next one.

\section*{Problem 2-5 Jones \& Pevzner, Problem 4.12}

In this problem, we have a long text string $T$, and a shorter pattern string $s$, we are trying to find the first occurence of $s$ in $T$, if there is any.

We suppose that the length of $T$ is $m$, and the length of $s$ is $n$.

\subsection*{Pseudocode}

\begin{algorithm}[H]
\caption{Pseudocode for finding pattern}
\begin{algorithmic}
\Function{FindPattern}{T,s,m,n}
 \While{$i<(m-n)$}
  \If{$s[i]==T[i]$}
    \For{$j$ in range $i\to (i+n)$}
     \If{$s[j] == T[j]$}
      \If{$j!=(i+n)$}
       \State $j++$
      \EndIf
      \If{$j==(i+n)$}
       \State Return $s[i:(i+n)]$
      \EndIf
     \EndIf
     \If{$s[j]!=T[j]$}
      \State $i = j$
      \State Exit for loop
     \EndIf
    \EndFor
  \EndIf
  \If{$s[i]!=T[i]$}
   \State $i++$
  \EndIf
 \EndWhile
\EndFunction
\end{algorithmic}
\end{algorithm}

\subsection*{Time Complexity}

In our implementation, we will iterate over the long string with our pattern. If we find that the first letter of our pattern is equal to the position we are checking in the longer text string, we will continue to check the following positions of our pattern to see if they match the ones on the long string. 

Then in worst case scenario, we will have to check if the pattern is in the long string, on every position possible of the long string, without success. Then the time complexity of our algorithm will be $O((m-n)*n)$.

\section*{Problem 2-6 Jones \& Pevzner, Problem 4.16}

The $YETANOTHERMOTIFSEARCH$ algorithm is similar to $BRANCHANDBOUNDMOTIFSEARCH$, as the algorithm, tries to find the best score by trying only the motifs that may give us a better score than the current best score.

The main difference between the two algorithms, is that for building all the possible motifs, $YETANOTHERMOTIFSEARCH$ uses recursive calls to the $FIND$ function, while $BRANCHANDBOUNDMOTIFSEARCH$, models the problem as a tree, and uses branch and bound for the same purpose.

\section*{Problem 2-7 Rosalind}

This problem was solved entirely on \textbf{Rosalind}.

\textbf{Username:} \textit{jarechalde}

\end{document}