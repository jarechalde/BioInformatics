\documentclass{article}
\usepackage{listings}
\usepackage{amsmath}
\usepackage{graphicx}
\usepackage{float}
\usepackage{subcaption}
\usepackage[linewidth=1pt]{mdframed}
\usepackage[colorlinks]{hyperref}

\usepackage{algorithm}
\usepackage{algpseudocode}

\usepackage{verbatim}

\hypersetup{citecolor=DeepPink4}
\hypersetup{linkcolor=DarkRed}
\hypersetup{urlcolor=Blue}

\usepackage{cleveref}

\setlength{\parindent}{1em}
\setlength{\parskip}{1em}
\renewcommand{\baselinestretch}{1.0}

\begin{document}

\begin{titlepage}
	\centering
	{\scshape\LARGE Assignment 3\par}
	\vspace{1cm}
	{\scshape\Large Bioinformatics (CIS 455)\par}
	\vspace{1.5cm}
	{\Large\itshape Javier Arechalde\par}
	\vfill
	{\large \today\par}
\end{titlepage}

\section*{Problem 3-1 Jones \& Pevzner, Problem 5.1}

\subsection*{Correct answer:}

$$A(\pi) \geq OPT(\pi)/App_{ratio}$$
$$12 \geq OPT(\pi)/\frac{1}{4}$$
$$OPT(\pi) \leq 3$$

\subsection*{What if A was a minimization algorithm?}

$$App_{ratio} \geq A(\pi)/OPT(\pi)$$
$$\frac{1}{4} \geq 12/OPT(\pi)$$
$$OPT(\pi) \geq 48$$

\section*{Problem 3-2 Jones \& Pevzner, Problem 5.4}

\subsection*{1.}

We are going to perform \textit{ImprovedBreakpointReversalSort} on $\pi$ = 3 4 6 5 8 1 7 2.

We start by adding 0 and 9 to the sequence.

\begin{center}
0 3 4 6 5 8 1 7 2 9 $b(\pi) = 7$

0 3 4 \textbf{6 5} 8 1 7 2 9

0 3 4 5 6 8 1 7 2 9 $b(\pi) = 6$

0 3 4 5 6 \textbf{8 1 7} 2 9

0 3 4 5 6 7 1 8 2 9 $b(\pi) = 5$

0 3 4 5 6 7 1 \textbf{8 2} 9

0 3 4 5 6 7 1 2 8 9 $b(\pi) = 3$
\end{center}

As there are not any decreasing strips, we flip an increasing one.

\begin{center}
0 3 4 5 6 7 \textbf{1 2} 8 9

0 3 4 5 6 7 2 1 8 9 $b(\pi) = 3$

0 \textbf{3 4 5 6 7 2 1} 8 9

0 1 2 7 6 5 4 3 8 9 $b(\pi) = 2$
\end{center}

Finally,

\begin{center}
0 1 2 \textbf{7 6 5 4 3} 8 9

0 1 2 3 4 5 6 7 8 9 $b(\pi) = 0$
\end{center}

\subsection*{2.}

The permutation that used the if test was this one:

\begin{center}
0 3 4 5 6 7 1 2 8 9

0 3 4 5 6 7 \textbf{1 2} 8 9

0 3 4 5 6 7 2 1 8 9
\end{center}

\subsection*{3.}

The sequence shown below uses 4 reversals instead of the 5 reversals used by \textit{ImprovedBreakpointReversalSort}.

\begin{center}
0 3 4 5 6 8 1 7 2 9

0 3 4 5 6 \textbf{8 1 7 2} 9

0 3 4 5 6 2 7 1 8 9

0 \textbf{3 4 5 6} 2 7 1 8 9

0 6 5 4 3 2 7 1 8 9

0 6 5 4 3 2 \textbf{7 1} 8 9

0 6 5 4 3 2 1 7 8 9

0 \textbf{6 5 4 3 2 1} 7 8 9

0 1 2 3 4 5 6 7 8 9
\end{center}

\section*{Problem 3-3 Jones \& Pevzner, Problem 5.5}

\begin{center}
0 1 4 5 2 3 6 $b(\pi) = 3$
\end{center}

We choose the following permutation:

\begin{center}
0 1 4 \textbf{5 2 3} 6

0 1 4 3 2 5 6 $b(\pi) = 2$
\end{center}

Finally,

\begin{center}
0 1 \textbf{4 3 2} 5 6

0 1 2 3 4 5 6 $b(\pi) = 0$
\end{center}

\section*{Problem 3-4 Jones \& Pevzner, Problem 5.13}

\subsection*{$\pi_1$ and $\pi_2$}

There are 3 breakpoints between $\pi_1$ and $\pi_2$: 12, 24, 35.

\subsection*{$\pi_1$ and $\pi_3$}

There are 2 breakpoints between $\pi_1$ and $\pi_3$: 24, 35.

\subsection*{$\pi_2$ and $\pi_3$}

There are 2 breakpoints between $\pi_2$ and $\pi_3$: 14, 25.

\section*{Problem 3-5 Jones \& Pevzner, Problem 6.4}

\begin{algorithm}[H]
\caption{Implementation}
\begin{algorithmic}[1]
\Function{DPCHANGE}{M,c,d}
 \State $bestNumCoins_0 \leftarrow 0$
 \State $bestCoins_0 \leftarrow \{\}$
 \For{$m \leftarrow 1$  to $M$}
 \State $bestNumCoins_M \leftarrow \infty$
  \For{$i \leftarrow 1$ to $d$}
   \If{$m \geq c_i$}
    \If{$bestNumCoins_{m-c_i} + 1 < bestNumCoins_m$}
     \State $bestNumCoins_m \leftarrow bestNumCoins_{m-c_i} + 1$
     \State $bestCoins_m = \{bestCoins_{m-c_i},c_i\}$
    \EndIf
   \EndIf
  \EndFor
 \EndFor
 \State \Return $bestNumCoins_M$
\EndFunction
\end{algorithmic}
\end{algorithm}

\section*{Problem 3-6 Rosalind}

My \textit{Rosalind} username is \textbf{jarechalde}.

\section*{Problem 3-7 Rosalind}

\subsection*{afoster3}

What I like about this solution is that is really compact, and in a few lines can implement the whole solution.

\subsection*{dennison\_david}

What I like about this solution is that rather than using a for loop and going letter by letter, it uses a pointer and a while loop instead, breaking this loop when no more matches are found.

\subsection*{Schavan}

What I like about this solution is that it has a really good format, and it is also commented.

\end{document}
